\chapterno{3}
\chapter{Platinum Oxide Clusters }
\section{Introduction}
\begin{figure}
  \label{fig:myfig}
\image{flowchart_GOFEE.png}{width="90%"}
\caption{}
\end{figure}
\section{Experimental Observations on CO Oxidation}
\begin{figure}
  \label{fig:myfig}
\image{Pt7_co_conversion_expt.png}{width="90%"}
\caption{}
\end{figure}
\section{Low Energy Structures for Pt7Ox (x=0,4,10) + CO }
\begin{figure}
  \label{fig:myfig}
\image{Pt7_O0_CO.png}{width="90%"}
\caption{}
\end{figure}
\begin{figure}
  \label{fig:myfig}
\image{Pt7_O4_CO.png}{width="90%"}
\caption{}
\end{figure}
\begin{figure}
  \label{fig:myfig}
\image{Pt7_O10_CO.png}{width="90%"}
\caption{}
\end{figure}
\section{Formation and Free energies of Pt Oxides }
\begin{figure}
  \label{fig:myfig}
\image{formationenergies_PtOxides_new.png}{width="90%"}
\caption{}
\end{figure}
\begin{figure}
  \label{fig:myfig}
\image{free_energy_diagram_Pt7Oxides_750K.png}{width="90%"}
\caption{}
\end{figure}
\section{Adsorption energy of CO on Pt Oxide }
\begin{figure}
  \label{fig:myfig}
\image{Adsorption_energies_PtOxides.png}{width="90%"}
\caption{}
\end{figure}
\section{Formation energy of O2 and adsorption energy of CO on Pt7 Oxide }
\begin{figure}
  \label{fig:myfig}
\image{COEads_CoEchem_O2Ef_Pt7Oxides.png}{width="90%"}
\caption{}
\end{figure}
\section{Conclusion}
