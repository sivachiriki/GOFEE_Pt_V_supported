\chapterno{1}
\chapter{RM Based Feature Space Minimization as Crossover Operation for EA}
\section{Introduction}
\section{Application to  Surface Reconstruction of anatase-TiO$_2$ (001)}
\begin{figure}
  \label{fig:resultsforTiO2}
\image{results_TiO22l3l4l.pdf}
\caption{Cumulative success rate for a) two layer, b) three layer c) four layer anatase TiO2 systems.}
\end{figure}
\subsection{Feature Space Representation of TiO2 (001)  Surfaces}
\begin{figure}
  \label{fig:featureTiO2}
\image{feature_repre_TiO2_001surface.pdf}
\caption{ a) two layer b) three layer c) four layer global minimum structures}
\end{figure}
\section{Application to SiO$_2$ Crystal Structures (Crystobalite Type)}
\subsection{Crystobalite  SiO2 crystal structure from Literature}
\begin{figure}
  \label{fig:myfig}
\image{Crystobalite_SiO2_literature_topsideview.png}
\caption{Crystobalite type SiO2 crystal (top  and side view)}
\end{figure}
\subsection{Optimization of Cell Paramters}

\subsection{Feature Space Representation of SiO2 crystal structures}
\begin{figure}
  \label{fig:myfig}
\image{Featurespace_SiO2_crystalstru.pdf}
\caption{ a) random initial structure b) SiO2 crystobalite (literature) c) RM predicted structure}
\end{figure}
\subsection{Comparison between RM predicted and Literature Structure }
\begin{figure}
  \label{fig:myfig}
\image{Literature_RMpredicted_topview.png}
\image{Literature_RMpredicted_sideview.png}
\caption{Top and side view of both structures.}
\end{figure}
