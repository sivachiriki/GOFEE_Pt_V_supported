\chapterno{2}
\chapter{Water Splitting Reaction on Supported Vanadium Oxide Clusters $V_xO_y$ /TiO$_2$(101)}
\section{Introduction}
\section{Experimental Work}
\subsection{STM images of Clean TiO2 (101) Surface}
\subsection{STM images of $V_xO_y$/TiO2 (101) Clusters}
\begin{figure}
  \label{fig:myfig}
\image{Expt_V2O5_TiO2_101.png}{width="90%"}
\caption{Preparation of V2O5 Clusters on TiO2(101) Surface}
\end{figure}
\subsection{Experimental Observations Made by STM Images}
\begin{figure}
  \label{fig:myfig}
\image{Expt_V2O5_nH2O_TiO2_101_obser.png}{width="90%"}
\caption{Preparation of V2O5 Clusters on TiO2(101) Surface}
\end{figure}
\section{Global Structure Search}
\subsection{Methodology}
\subsection{Most Stable Structures for VO2, V2O4, V2O5}
\subsection{Most Stable Structures at Water Exposure }
\begin{figure}
  \label{fig:myfig}
\image{V2O5_2H2O_TiO2_lowlying.png}{width="90%"}
\caption{Low-lying isomers of V2O5 + 2 H2O Clusters on TiO2 (101)}
\end{figure}
\subsection{Most Stable Structures After Water Exposure}
\begin{figure}
  \label{fig:myfig}
\image{V2O5_H2O_TiO2_lowlying.png}{width="90%"}
\caption{Low-lying isomers of V2O5 + 2 H2O Clusters on TiO2 (101)}
\end{figure}
\subsection{Structures that Matches STEM Images}
\begin{figure}
  \label{fig:myfig}
\image{Expt_theory_V2O5_nH2O_TiO2_101_obser.png}{width="90%"}
\caption{Structures found during GO search were matches to the STEM images}
\end{figure}
\section{Binding Energy of Water On V$_x$O$_y$TiO2 (101) Clusters}
\begin{figure}
  \label{fig:myfig}
\image{Binding_energies_V2O5_nH2O.png}{width="90%"}
\caption{Binding Energies of low-lying isomers}
\end{figure}
\section{Density of States for Most stable structures}
\begin{figure}
  \label{fig:myfig}
\image{Total_DOS_TiO2_s_r_surface.pdf}{width="90%"}
\caption{Density of states for stoichiometric surface and reduced surface}
\end{figure}
\begin{figure}
  \label{fig:myfig}
\image{Total_DOS_V2O5_2H2O_lowlying.png}{width="90%"}
\caption{Density of states for 3 different structures with Ov }
\end{figure}
\section{Effect of Oxygen vacancy }
\begin{figure}
  \label{fig:myfig}
\image{BE_V2O5.png}{width="90%"}
\caption{}
\end{figure}
